\documentclass{article}

\begin{document}





\section{Komentarai}


Sioje sekcijoje parsinami tik komentarai:
% LaTeX Dokumento parsinime svarbiausia yra apeiti komentarus
% ir verbatimines aplinkas.
% Inline komentaras iprastiniame LaTeX dokumente prasideda % ir baigiasi, \n
       % kartu surinkdamas sekancioje eiluteje visus trailing whitespace.
% Taciau komentaras kartu uzbaigia ir LaTeX'o komandos skaityma
t.y \LaTeX% Sis komentaras texo kompiliatoriui pasako, 
% kad \LaTeX yra pilna komanda
   tekstas. Ismetus komentara sintaksinis ekvivalentas turetu buti: 
\LateX tekstas arba \LateX{}tekstas  
% o ne \LaTeXtekstas
Taciau jei vietoj \LaTeX butu zodis pvz: nebeprisikiskia%
          kopusteliaudavome -- yra ekvivalentu zodziui be komentaro: 
nebeprisikiskiakopusteliaudavome

Todel salinant komentarus yra svarbu zinoti konteksta, kuriame komentaras yra. 
% Cia kitas komentaras

\section{Aktyvus simboliai}

Kaip suzinoti ar esamas simbolis turi tiesiogine prasme ar ne?

LaTeX kalboje standartineje modoje yra apibrezti 9
aktyvus simboliai % {'#','$','%','^','{','}','_','~','\\'}
. Juos vadinsime metacharakter'iais. Visa kas neaktyvuota vienu is siu 
simboliu yra traktuojama, kaip tekstas. 

LaTeX'o kalboje nebuna persiklojanciu aplinku. %[1{ 12  ]2 }.
Aplinkos gali buti tiktai nested tipo. %[1{2 (3)}  ]
Sis skriptas turetu parsinti keliais lygmenym. Pirmuoju (virsutiniu lygmeniu) 
vadinsime isorinius elementus: 

1) neaktyvuota teksta

2) inline komentarus

3) zinomas komandas:
* komandas kartu su argumentais, jei zinomas ju paternas % \texttt{ARGUMENTAS}
* inline verbatimo komanda % \verb!tai verbatimas!
* grieztus switchus ir turini tarp ju %\iffalse TURINYS \fi
* simbolius % \\textbullet

* negrieztus switchus, kurie veikia tik esamoje aplinkoje %\tt \it
* TeXo opciju perapibrezimus %\tablewidth=10cm
* enviromentus

4) Su nezinomom komandom susidorosime rinkdami ju apubrezimus ir atskirai
surinkdami visus aktyviais skliaustais %{} 
apgaubtus elementus.

Rinkdami grieztus switchus bei enviromentus turime atsizvelgti,
kad ju viduje, gali buti apibrezta, kita sintakse. 
T.y. gali buti visai kitas metacharackteriu rinkinys ir pasikeitusi ju prasme.
% begin{listing} viduje % nebereiskia komentaro pradzios \end{listing}

% \iffalse viduje isvis neskaitomi jokie metacharai iskyrus '\',
% kad galetume pagauti \fi

Kad susidoroti su sia problema, nusprendziau aprasyti dinamine LaTeX'o
sintakse. T.y. pradedant darba metacharu 
rinkinys, % METACHARACTERS={'#','$','%','^','{','}','_','~','\\'}
komanda aktyvuojantis simbolis, % START_OF_CMD={'\\':None}
ilgos komandos apibrezimo simboliai, % COMMAND_CHARS=set(string.ascii_letters)
inline komentara aktyvuojantis simbolis, % COMMENT={'%':('\n',('w',))}
argumentu skirtukai, % ARG_DELIMS={'{':'}'}
yra standartiskai apibrezti, su galimybe parsinimo eigoje juos keisti.

Neisimtis yra ir komandu argumentai, 
% Pvz \index argumente '%' nera komentaras,
% kitas pavyzdys butu xymatrix{...}
taciau su jais susidoroti yra lengviau, nes ju argumentai
TIKRIAUSIAI visada buna apskliausti metaskliaustais % {}

Taciau, kitu komandu argumentuose gali buti komentaru
% \textbf{Tai yra tekstas % o tai nera argumento uzbaigimas}

Taigi renkant argumentus BUTINA zinoti ar argumente nera pakeista 
iprastine sintakse






\section{Komandos}




\subsection{Kaip atpazinti komandos pradzia}

Norint nuskaityti komandos identifikacija reikia atpazinti simboli,
kuris reiskia jos pradzia. 
Iprastoje \LaTeX 'o modoje komanda aktyvuojantis simbolis 
yra % \ 
Toliau einanti ascii raidziu seka yra komandos
pavadinimas. Taciau jei sekantis simbolis nera nei viena is ascii 
raidziu, tai komandos identifkacija ir bus tas simbolis.
Svabu teisingai pagauti komanda aktyvuojancius simbolius.

\\ \textbackslash \{ \} \\ 
\\\\\\ \\[10pt] {\ }\ 





\subsection{Komandos pavadinimo atpazinimas}

\textbackslash{}\textbf{komandos_pavadinimas}
\\\ \, \; \*
\
\\texttt{cia ne komanda} \quad \\qquad
\\\\\\\\\\\ \\\
\textbullet





 
\subsection{Inline verbatimai}



Renkant komandu argumentus pries tai yra svarbu atpazinti, 
kur yra komentaras ir kur verbatimine aplinka.
Inline verbatimai skiriasi nuo kitu standartiniu LaTeX komandu tuo, kad 
ju argumento skirtukai yra bet kokie nealfabetiniai simboliai.
Jo argumento viduje nera jokiu aktyviu simboliu.

\verb!Tai yra verbatimas!
\verb
verbatimo skirtukais gali buti net newline
\verb!aktyvus charai cia tik simboliai {}#$%^_~\\!
\verb# verbatimo viduje % reiskia procenta#
\verb% verbatimas gali buti 
ir daugialinijinis% 





%%%%%%%%%%%%%%%%%%%%%%%%%%%%%%%%%%%%%%%% 
\subsection
%%%%%%%%%%%%%%%%%%%%%%%%%%%%%%%%%%%%%%%%
%%%%%%%%%%%%%%%%%%%%%%%%%%%%%%%%%%%%%%%% 
       {Kelias iki argumento}
%%%%%%%%%%%%%%%%%%%%%%%%%%%%%%%%%%%%%%%% 

Texas sutikes komanda turincia argumentus ignoruoja 
komentarus ir whitespace iki argumento pradzios.

\texttt               {tekstas}

\textit 
            {tekstas2}

\textbf           
%%%%%%%%%%%%%%%%%%%%%%%
%%%%%%%%%%%%%%%
           {textas3}

\def%%%%%%%%%%%%%%%%%%%                       
         \komanda{komanda} 

$\frac           1     2$



\subsection{Problema su argumentais}

Kadangi mus domina suparsinto medzio susirinkimas,
o ne dvi failo kurimas realiu laiku interpretuojant skripta,
uzduotis siek tiek pasunkeja. 
Kad ir kaip bebutu keista, skriptas noredamas surinkti 
komandos argumenta turi jau moketi surinkti enviromentus,
grieztus switchus ir kitas komandas su argumentais.
Puikus to pavyzdziai:
\textbf{
\begin{comment}
tai netikras galas}
\end{comment}
}

\texttt{\iffalse }\fi }

\textit{ \index{a%b} }

\textbf{lalala}


Taigi pries renkant globaliai komandu argumentus
turi buti didele duomenu baze su surasytom aplinkom:
% 
\begin{enumerate}
\item kuriu turinys yra komentarai
  % 
\begin{enumerate}
\item[a] \verb!comment! paketas 
\begin{itemize}
\item \verb!\comment...\endcomment!
\item \verb!\begin{comment}...\end{comment}
\item 
  % 
\begin{verbatim}
\includecomment{versiona}
\excludecoment{versionb}
\end{verbatim}
  % 
Leidzia naudoti kaip komentarus naujus apibrezimus 
% 
\begin{verbatim}
\versiona ... \endversiona
\begin{versiona} ... \end{versiona}
\end{verbatim}
 % 
\end{itemize}
\end{enumerate}
 
\item kuriu turinyje egzistuoja kita negu standartine sintakse.
Konkreciu atveju, tos aplinkos, kuriu turinyje komentaro zenklas
reiskia visai ka kita.
\begin{itemize}
\item \verb!\index{aaa%bbb}!
\item \verb!\begin{listing}...\end{listing}!
\item \verb!\begin{Verbatim}[commentchar=@]...\end{Verbatim}!
\end{itemize}
  


\end{document}
